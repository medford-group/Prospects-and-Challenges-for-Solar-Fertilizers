\documentclass[journal=jacsat,manuscript=article]{achemso}



\usepackage[version=3]{mhchem} % Formula subscripts using \ce{}
\usepackage{amsmath}
\usepackage{graphicx}
\usepackage{wrapfig}
\usepackage{longtable}
\usepackage{placeins}
\usepackage[colorinlistoftodos]{todonotes}
%\usepackage[colorlinks=true, allcolors=blue]{hyperref}
\usepackage{subcaption}
\usepackage{comment}
\usepackage{totcount}
\usepackage{makecell}
\usepackage{lastpage}
\title{Prospects and Challenges for Solar Fertilizers Supporting Information}
\begin{document}
\renewcommand{\thepage}{S\arabic{page}}
\renewcommand{\theequation}{S\arabic{equation}}
\renewcommand{\thetable}{S\arabic{table}}
\renewcommand{\thefigure}{S\arabic{figure}}
\newpage
\regtotcounter{page}
\regtotcounter{table}
\regtotcounter{figure}
%\begin{suppinfo}
%\total{page}\ pages, \total{table}\ tables, \total{figure}\ figures
\listoftables
\listoffigures
\newpage

\section{Calculations}
%The concentration of NH$_3$ at which it can be detected through the human sense of smell
%\begin{align*}
%\mathrm{
%& 50ppm \: NH_3 \times \frac{10^{-6}}{ppm} \times 101.325kPa \times \frac{1000 \: Pa}{kPa} \times 0.59 \frac{mol}{m^3Pa} \times \frac{m^3M}{1000mol} \times \\
%& \frac{mM}{10^{-3} M} = 2.98mM \: NH_3 \approx 3mM \: NH_3
%}
%\end{align*}

The approximate per hectre nitrogen contributed to croplands by biological nitrogen fixation. total biological nitrogen fixation is estimated at 33 Tg N/yr by Smil\cite{Smil_1999_2} and cropland area estimated by FAOSTAT. All values for 1999
\begin{equation}
\mathrm{
33 \frac{Tg}{yr} \: \times \frac{1}{1540687.59 kha} \: \times \frac{kha}{1000 ha} \: \times \frac{10^{9} kg}{Tg} = 21.4 \frac{kg}{ha\:yr}
}


\end{equation}
Herridge et. al. \cite{Herridge_2008} estimate the yearly fixed nitrogen production to be 50-70 $\frac{Tg}{yr}$ translating to 45.4 and 32.4 $\frac{kg}{ha \: yr}$


Ammount of activated carbon needed to fertilizer one hectare of land for one year. Based off 50kg N/ha yr and highest adsorption capacity of NH$_3$ on activated carbon found in the literature (16mg NH$_3$/g C)\cite{Gon_alves_2011}.
\begin{equation}
\mathrm{ 100 \frac{kg \: N}{ha \: yr} \: \times \frac{g \: C}{16 \: mg \: NH_3} \: \times \frac{17g\:NH_3}{14g\:N}  \times \: 1000 \frac{mg}{g} = 7,589 \frac{kg\:C}{ha \: yr}
}
\end{equation}

The volume density of NH$_3$ in maximally saturated activated carbon. Highest adsorption capacity of NH$_3$ on activated carbon found in the literature (16mg NH$_3$/g C)\cite{Gon_alves_2011}. The density of activated carbon was assumed to be 400 kg/m$^3$
\begin{equation}
\mathrm{\frac{16 \: mg \: NH_3}{g \: C} \: \times \frac{0.4g \:C}{cm^3} \: \times \frac{1000 cm^3}{L}  \times \: \frac{mmol \: NH_3}{17 mg \: NH_3} = 376 \frac{mmol\:NH_3}{L}
}
\end{equation}

The energy per mole required to produce ammonia by electrocatalysis. Values from Song et al.\cite{Song_2018}. 0.0525 is the electrical energy efficiency they report
\begin{equation}
\mathrm{\frac{1}{0.0525} \times \frac{339 \: kJ}{mol} = 6460 kJ/mol
}
\end{equation}

The energy per mole required to produce ammonia by photocatalysis. Values from Shiraishi et. al.\cite{Shiraishi_2018}. 0.001 is the solar to chemical efficiency (defined as energy in formed ammonia/light energy added)
\begin{equation}
\mathrm{\frac{1}{0.001} \times \frac{339 \: kJ}{mol} = 3.39 \cdot 10^5 kJ/mol
}

\end{equation}
%\begin{equation}
%\mathrm{\frac{100 kg}{ha*yr} * \frac{1000 mol}{ 17 kg} * \frac{1 year}{365 days} * \frac{1 day }{ 24 hr} / ((100/10^6) \frac{55.41 mol}{L} to L/(ha*hr)
%}

%\end{equation}
%\newpage

\pagebreak
\bibliography{main.bib}
%\end{suppinfo}
\end{document}
