\section{Introduction}

% Motivate need for new process
The Haber-Bosch process for thermochemical synthesis of ammonia from nitrogen and hydrogen transformed the global fertilizer industry and is a critical enabler of the continued expansion of human populations \cite{Smil_1999}. The process is an impressive feat of modern chemical engineering, producing around 150 million tons of ammonia per year at a thermodynamic efficiency of up to 70\% \cite{Schloegl_2003,Schiffer_2017}. However, the process also has downsides. The scale of the process leads to massive energy consumption of 2.5 exajoule per year, and the hydrogen feedstock is typically obtained via methane reforming, leading to a carbon footprint of 340 Mt of CO$_2$ equivalent per year; this is the highest impact of any commodity chemical \cite{Schiffer_2017}. Furthermore, the high temperatures ($\sim$700 K) and pressures ($\sim$100 bar) lead to substantial capital cost and economies of scale, driving highly centralized production. This is in contrast to the decentralized use of ammonia-based fertilizers (Fig. \ref{fig:usemap}), which results in high transportation costs and additional emissions \cite{West_2002}. This is particularly impactful in remote locations such as sub-Saharan Africa, where soils are often nutrient-limited due to lack of access to fertilizers \cite{Gilbert_2012, Mueller_2012}. The opposite problem of over-fertilization also has a negative environmental impact in more developed nations due to the periodic application of highly concentrated fertilizers that cause nitrate pollution in waterways and vast ocean ``dead zones'' through leaching \cite{Diaz2008}. Finally, the intense conditions and reactive nature of concentrated ammonia based fertilizers lead to safety and national security concerns, as evidenced by explosions at fertilizer plants and the common use of fertilizers in makeshift explosives \cite{Marlair_2005}.

%\begin{figure}
%    \centering
%    \includegraphics[width=1\textwidth]{Figures/decentralization.jpg}
%    \caption{\hl{placeholder figure for illustration of decentralization. Overlay fertilizer use with locations of ammonia plants. Possibly include solar flux map.}}
%    \label{fig:usemap}
%\end{figure}

\begin{figure}
    \centering
    \includegraphics[width=1\textwidth]{Figures/solar_map.png}
    %\includegraphics[width=1\textwidth]{Figures/N_fertilizer_map.png}
    \includegraphics[width=1\textwidth]{Figures/croplands_map.png}
    \caption{(top)Horizontal solar radiation intensity over the surface of the earth averaged ($kW/m^2$) (bottom) percent of land dedicated to crop production (\%)}
    \label{fig:usemap}
\end{figure}

%http://sedac.ciesin.columbia.edu/data/sets/browse?facets=theme:agriculture
%http://globalsolaratlas.info/downloads/world

% Motivate solar fertilizers
One possible strategy to overcome these disadvantages is decentralized production of fertilizers using solar energy. These ``solar fertilizers'' provide a route to harness solar energy, nitrogen, and water/oxygen from the air to produce lower-concentration ammonia- or nitrate-based fertilizers at or near the point of use. This is advantageous from the perspective of solar energy capture since the intermittent energy is directly captured in a storable product that can be utilized near the point of production, avoiding issues with electricity storage and transport \cite{MacKay_2013}. There are also advantages from an agricultural perspective, since inexpensive feedstocks and reduced transport costs may significantly improve access to fertilizers in remote and developing regions, and the use of lower-concentration fertilizers may enable novel strategies of nutrient management that can reduce groundwater pollution. Further, low-concentration fertilizers produced at ambient conditions are inherently safer both from the perspective of the process and the product. Although fertilizers require a range of nutrients including N, P, and K, we focus on N fertilizers in this work since it is often the limiting nutrient\cite{Yousaf2017}. Additionally, in developing regions such as Africa, there exists an overabundance of soil P fertilization with respect to N\cite{VanderVelde2014}, limiting crop productivity. %and the production is the most centralized (I think this statement is false, ) \needcite.% the key challenge lies in the fixation of nitrogen and hence this work is focused solely on nitrogen-based fertilizers. 
%Development of solar-driven nitrogen fixation would likely serve as a driver for the development of solar technologies for other nutrients.
%\hl{This is a little awkward, but we need to point out somewhere that we are only considering N2 fixation here. I'm not sure we even need to address the P and K fertilizers. mch- I think this is OK, the last sentence regarding motivating other solar nutrient technologies might not be needed though}

% Motivate need for collaboration with agriculture/fertilizer
Solar fertilizers hold substantial promise as a route to solar energy capture and sustainable agriculture, but there are also considerable challenges. One critical challenge is in the development of a viable strategy for efficiently using solar energy to dissociate the strong dinitrogen triple bond at ambient conditions. Nitrogen fixation at ambient conditions is a key objective of chemistry, and has been the subject of considerable research in homogeneous catalysis, enzyme catalysis, and bioengineering, yet no viable strategies have emerged due to issues with low conversion and/or stability under realistic conditions \cite{Vicente_2017,Bur_n_2017,MacLeod_2013,Foster2018}. %need more citations here.
More recently there has been a surge of interest in photo- and electrocatalytic nitrogen fixation by heterogeneous catalysts \cite{Medford_2017,Kyriakou_2017,Foster_2018,Chen_2018}. This route is particularly interesting from the perspective of solar fertilizers since photoelectrochemical systems interface well with solar energy, have the potential to scale relatively easily, and have been the subject of considerable research in the solar fuels community \cite{Kondratenko2013}. However, further work is needed to improve the yield and efficiency of photo- and electrochemical nitrogen fixation. %In this work we focus on photo(electro)chemical routes to solar fertilizer production and seek to identify targets and strategies for future research.

Solar fertilizers are also expected to differ significantly from traditional fertilizers, opening a range of additional agronomic challenges and opportunities. One key difference is that solar fertilizers are expected to have considerably lower fixed nitrogen concentrations, owing to the fact that photo(electro)chemical nitrogen fixation efficiencies are unlikely to compete with the 70\% efficiency of the Haber-Bosch process \cite{Schloegl_2003, Singh_2017}. %cite selectivity challenge papers?
Separating and concentrating the ammonia will require additional steps that add complexity to the process and require energy. Direct utilization of dilute fertilizers will avoid or reduce the amount of energy needed for separation/concentration, and may also enable more controlled nutrient management. However, this represents a paradigm shift in agricultural practice, and considerable effort is needed to understand how dilute solar fertilizers can be sustainably and practically integrated with agricultural systems. These considerations will also inform the development of the photo(electro)catalytic processes for solar fertilizer production, and hence should be considered in parallel.

In this work we identify key considerations and performance targets for the photo(electro)chemical production of dilute solar fertilizers from the perspective of catalysis and agronomics. Some specific advantages and disadvantages of dilute and decentralized fertilizer production are outlined and the potential agronomic use cases and impacts are examined. Some specific strategies for the design of a reaction/separation process for photo(electro)chemical fertilizer production are considered, and assessed qualitatively on the bases of metrics that will impact their ultimate viability. These possible designs are used along with back-of-the-envelope calculations to quantify initial performance targets and limiting cases for catalyst reactivity and separation efficiency and suggest specific materials properties and tests that will inform process design. We hope that these considerations will serve as a foundation and guide for future research in the development of photo(electro)chemical processes for solar fertilizer production.

\section{Agronomics of solar fertilizers}

Current fertilizer production relies on the Haber-Bosch process, which is highly centralized with fewer than 100 total production plants \cite{McArthur_2017} for over 500 million total farms \cite{FAO_2014,Lowder_2016}. The centralization is driven primarily by the harsh reaction conditions of the process. The high temperature ($\sim$450 $^\circ$ C) and particularly pressure ($\sim$ 20 bar) of the process lead to a favorable economy of scale, with a typical scaling factor of 0.7 \cite{Ullmann_amm_2006}. This incentivizes high-volume production and high capital investment, with most plants operating at a capacity greater than 1200 tonne day$^-1$ with a typical capital cost of \hl{???} \cite{Ullmann_amm_2006}. This leads to long payback periods, and encourages development of production facilities in regions with stable access to the natural gas feedstock and reliable infrastructure \cite{McArthur_2017}, as indicated by the presence of ammonia production primarily in the developed world (see Fig. \ref{fig:map}). The resulting products must be transported to agricultural centers, leading to fertilizer products with highly concentrated nutrient density. This section briefly explores the implications of the centralized Haber-Bosch process on the economics of fertilizer production, and explores the potential impact and challenges of decentralized production of dilute fertilizers in both developed and developing regions.

\subsection{Decentralization of fertilizer production}
\label{sec:decentralized}

The cost of fertilizers can be broken down into production, transportation, and storage. The production cost of fertilizer is controlled primarily by the cost of natural gas used to produce hydrogen for the Haber-Bosch process, and hence varies with geographic, economic, and geopolitical factors \cite{Huang2007,Etienne2016}. 
This leads to variable and uncertain cost of fertilizers and presents challenges in agricultural planning \cite{Etienne2016}. Furthermore, the cost of transportation is highly variable depending on location. Barge, pipeline, and rail transport are normally used for long-distance anhydrous ammonia transportation, while trucks are preferred for short distances. Distance, location of plant site relative to the agricultural area, availability of transportation equipment, and relative cost of available carriers are the major governing factors for selection of a typical anhydrous ammonia transportation system. Typical costs reported for long distance (greater than 1600 km) pipeline, barge, and rail transport are \$0.0153, \$0.0161 and \$0.0215 per ton per kilometer, respectively \cite{ammonia_encyclopedia}.  Short truck transportation costs are expected to be much higher. For distances of the order of 100 km, typical costs of \$0.0365 per ton per kilometer are reported \needcite. Overall, international shipping of ammonia between the United States and Western Europe costs on the order of \$ 35/t \needcite. Additionally, storage costs must be considered due to the cyclic nature of agricultural ammonia consumption caused by the harvest seasons. It has been reported that roughly 75\% of the dedicated fertilizer production is sold during the harvest season \needcite. To reduce storage costs resulting from this cyclic consumption pattern, large refrigerated anhydrous ammonia storage vessels are used, which add another \$11-80/t to ammonia cost \cite{IFDC_1998,ammonia_encyclopedia}. Thus, the freight costs can account for more than half of the delivered cost of ammonia in some countries \needcite. The variation in ammonia production costs in different countries is presented in Figure \ref{fig:relative_costs} \cite{maxwell2004synthetic}, showing that in many regions outside the US the cost of gas feedstocks accounts for less than half of the total cost. The hazardous nature of ammonia also leads to challenges with transportation and storage, particularly in regions with poor infrastructure \cite{Etienne2016}. 

\begin{figure}
    \centering
    \includegraphics[width=0.5\textwidth]{Figures/Cost_By_location.png}
    \caption{Data from Maxwell 2004\cite{maxwell2004synthetic}, }
    \label{fig:relative_costs}
\end{figure}

The production, transportation, and storage costs are the main components of fertilizer price, but the overall cost is not directly derived from these categories. For example, the price of fertilizer in Thailand (\$ 287/ton) is roughly half the price of fertilizer in Mozambique (\$ 567/ton), but this difference cannot be attributed directly to any single category. One critical factor that controls this overall price is economy of scale. Developing nations in Africa are often purchasing smaller quantities of fertilizer from the international market, leading to a limited ability to bargain for lower wholesale prices \cite{Wanzala2013}. Larger agricultural markets such as Asia are able to more effectively distribute fixed costs of transportation and negotiation across more units of fertilizer, translating to lower prices at the farm scale. This scale-up is not possible in less developed markets for a variety of reasons including port capacity and poor transportation infrastructure \needcite. Political instability often compounds this problem by causing existing infrastructure to deteriorate due to lack of use\cite{Foster_2011}. Less developed markets are also subject to more uncertain demand owing to lack of access to capital by smallholder farmers and unpredictable implementation of government subsidies \needcite. Overall, these factors lead to the perverse situation in which fertilizers are most expensive in the poorest places where the need is greatest. This is a key factor in the distressing fact that despite the tremendous technological developments of the recent decades, world hunger is currently increasing with over 800 million people suffering from undernourishment as of 2016 \cite{FAO_2017}. Notably, many of these economic and geopolitical factors could be alleviated by decentralized production of dilute fertilizers from renewable resources. Lack of dependence on natural gas would reduce volatility in fertilizer production costs \cite{Etienne2016}, and producing fertilizer at or near the point of use would reduce transportation costs and reduce the price dependence on economies of scale  \cite{IFDC_1998}. Furthermore, local production would improve certainty in fertilizer availability and reduce the influence of external factors such as tariffs and subsidies. Agricultural production has also been tied to general economic prosperity, and local fertilizer manufacturing industries in developing countries could spur substantial economic growth \cite{McArthur_2017}.
%For example, transportation costs as low as \$0.0043 per ton per km for aqueous ammonia have been reported for China \cite{guo2002animal}.

\begin{figure}
    \centering
    \includegraphics[width=1\textwidth]{Figures/Cost_Breakdown.pdf}
    \caption{The price breakdown for fertilizer in (a) Thailand and (b) Mali for the year 2013. Relative areas reflect the ratio of costs in the two countries (\$282 in Thailand and  \$509 in Mali) \cite{Wanzala2013} \hl{we can select different countries and/or add more. Would be interesting to include the US.}}
    \label{fig:cost_pies}
\end{figure}



%\begin{figure}
 %   \centering
 %   \includegraphics[width=0.7\textwidth]{Figures/gas_vs_fert.pdf}
 %   \caption{The cost of natural gas paid by industrial consumers in the United States and the price index of fertilizer between over time (www.eia.gov and usda.gov)}
%    \label{fig:gas_vs_fert}
%\end{figure}

%The concept of decentralization is difficult to quantify in general \cite{Schneider_2003}, but in this context we propose the decentralization should scale directly with the total number of fertilizer production facilities. It is useful to examine order-of-magnitude estimates of these quantities to assess the prospects of decentralization. A recent report identified a total of 63 major fertilizer production facilities based on data from the 10 largest producers of ammonia-based fertilizers \cite{McArthur_2017}. While this list may be incomplete, it is expected to be on the correct order of magnitude. A typical 1000 tpd ammonia plant can meet the ammonia requirements of many small countries. At a medium application rate of 40 kg N/ha, a typical ammonia plant can fertilize 7.5 million hectares of land. Fertilizer production on this scale is not available in many developing countries; logistics, economic and political issues constrain construction of large scale ammonia plants in such regions.\cite{IFDC_1998} 
%For landlocked countries that are located at a long distance from the ports, the freight cost doubles or even triples the cost of ammonia-based products, and thus limits consumption.

%This should be normalized by the total agricultural output, which we propose can be estimated by the total number of farms or the total amount of arable land. Both numbers are difficult to know exactly, but a 2014 report estimated the total number of farms to be in excess of 570 million with a total area of around 4.9 billion hectares (another FAO document said 1.5 billion ha) as of 2010 \cite{FAO_2014,Lowder_2016}. This indicates that a single ammonia-based fertilizer plant serves on the order of 10 million farms, or 100 million \hl{(24 million)} hectares of arable land (see Figure \ref{fig:map}) \hl{We need to refine these estimates and ensure that the 4.9 billion number does not include grazing land. The map is based off the smaller number of 24 million based on this source: https://ourworldindata.org/yields-and-land-use-in-agriculture.}.
%\begin{figure}
%    \centering
%    \includegraphics{Figures/approx_size.pdf}
%    \caption{The approximate area of land fertilized by an average single Haber-Bosch plant based on rough estimates above. \hl{Revise as needed based on data estimates.}}
%    \label{fig:map}
%\end{figure}


There are a continuum of options for moving away from this highly centralized scenario. The most extreme alternative would be fully decentralized farm scale fertilizer production, and this would have the largest societal and agricultural impact if deployed in low-income countries where access to fertilizer is limited. There is a large range of farm sizes, but the majority of farms in low-income countries are $\lessapprox$  1 ha \cite{Lowder_2016}. Fertilizer production at the scale of small farms would  correspond to roughly 1 fertilizer production facility per $\sim$1 ha, an increase of $\sim$ 7 orders of magnitude in the total number of fertilizer production facilities. This would also correspond to proportional decrease in the scale of production.
The global average nutrient load needed for fertilization is $\sim$ 50 kg-N/(ha yr) \needcite, which
%\hl{FAO data mch- we may want to round up to 100, based on a discussion with IFDC, I think 50 kg-N/ha is due to disparity}, 
 can be used to estimate the annual budget for on-site fertilizer production based on the cost of fertilizer per country. For example, the price of urea in Ghana in 1999 was one of the highest, approximately \$1100/MT, or \$ 2350/MT-N (urea is 46\% N by weight). \needcite \hl{Would be great to get some more recent numbers here}
 This corresponds to an expected annual fertilizer budget on the order of \$ 120/year. This modest number suggests that decentralized fertilizer production at the scale of small farms must have very low capital and operating costs, even in countries where the cost of fertilizer is very high. Furthermore, the process must be sufficiently robust that specialized operators are not needed for operations or maintenance, and additional constraints such as water usage and fertilization infrastructure (e.g. irrigation) must be considered. However, the current economic situation in many developing countries leads farmers to use little to no fertilizers \needcite, indicating that even the production of an inferior fertilizer product can impact agricultural output if the cost is sufficiently low and infrastructural requirements are minimal. This suggests that ``frugal innovation'' strategies \cite{Weyrauch_2016} may be required for the development of inexpensive and robust processes for producing fertilizer at the scale of small farms in the developing world.
 
An alternative approach to farm-scale production is to target larger farms ($\gtrapprox$ 100 ha), particularly those with irrigation systems in place. These larger farms are more common in developed countries \cite{Lowder_2016}, which presents an economic challenge for decentralized production since it will be competing more directly with traditional fertilizers. For example, the cost of urea in the United States is approximately \$200/MT, or \$435/MT-N as of 2018 \needcite. This is offset by the larger capacity of the farms, and heavier fertilization in the developed countries. Assuming a nutrient load of $\sim$ 100 kg-N/(ha yr) leads to an approximate annual fertilizer production of 10 MT-N/yr budget of $\sim$ \$4,350 per year \hl{It would be even better to get actual numbers for fertilizer budgets for e.g. citrus farms}. This number is relatively modest, but there are additional incentives for larger farms to invest in decentralized fertilization. These larger farms require larger capital investment, and the reduced volatility in price for fertilizers produced on site would improve the predictability of returns. The integration of on-site solar fertilizers with existing irrigation infrastructure may reduce the costs associated with delivering fertilizers to crops, or enable more efficient fertilizer utilization, as discussed further in Sec. \ref{sec:dilute}. There are also challenges for scaling solar fertilizers to larger farms. Solar fertilizer production will require a higher level of technological sophistication, particularly if electrochemical technologies are employed. These approaches will require installation, maintenance, and potentially operation by experts. It is unlikely that a full-time employee could be dedicated to fertilizer production, even at relatively large farms. Nonetheless, periodic access to experts for installation and maintenance is not an issue in developed regions, and numerous industries such as solar capture and home batteries operate on similar business models \needcite. This suggests that solar fertilizers are potentially viable for farm-scale production in developed areas as long as they can be operated with only periodic maintenance.

A third scenario is the production of solar fertilizers at a semi-centralized multi-farm scale. The challenge with more centralized production scenarios is more direct competition with the efficient and inexpensive Haber-Bosch process, since issues with transportation and distribution will be similar. Nonetheless, semi-centralized production can avoid costs and uncertainties associated with international or trans-marine distribution, and the lack of reliance on natural gas as a feedstock can reduce price volatility. Furthermore, coupling solar fertilizer production with the production of other agricultural products such as phosphorus, potassium, agricultural lime, or biochar can alleviate transportation issues by taking advantage of existing infrastructure. For example, a distributed network of fast pyrolysis facilities for simultaneous production of fuel and agricultural biochar has been suggested as a route to carbon-negative energy production \cite{Lehmann2007,Lehmann_2007, Glaser2002}. Coupling these fast pyrolysis plants with photo(electro)chemical nitrogen fixation presents a route to produce nutrient-enriched biochars, as discussed further in Sec. \ref{sec:dilute}. According to a technoeconomic analysis of fast pyrolysis these facilities would process on the order of 2000 MT of biomass per day, with a yield of $\sim$ 20\% biochar  \needcite \hl{https://www.nrel.gov/docs/fy11osti/46586.pdf}. This corresponds to around 150,000 MT-biochar/yr. The amount of biochar applied to farms varies widely from 0.5 - 50 MT/ha \cite{Galinato2011}, but 5 MT/ha is proposed as a reasonable number based on biochar uptake (see Sec. \ref{sec:dilute}) and prior agricultural studies \cite{Galinato2011}. This corresponds to $\sim$ 30,000 ha per facility, or 3000 MT-N/year (assuming 100 kg-N/ha-yr). Assuming the price of nitrogen nutrients is similar to that of urea in the developed world (\$435/MT-N) this leads to a substantial annual budget of \$1.3 million/facility for solar fertilizer production. This would lead to economic viability of more sophisticated solar fertilizer technologies that require full-time expert operation, such as high-temperature operation and/or large-scale solar concentrators. These semi-centralized approaches carry the largest infrastructural burden, and will face substantial challenges in implementation. However, approaches such as enriched biochar production as a byproduct of biofuel present exciting opportunities for simultaneously improving the sustainability of the agricultural and energy sectors through coupled infrastructural developments.

There are many other possible scenarios for solar fertilizer production, and the qualitative analysis above is far from complete. Yet, the order-of-magnitude estimates provided suggest that there are many routes through which solar fertilizers can compete with the established Haber-Bosch process by utilizing the advantages of decentralized production offered by photo(electro)chemical processes. Other niche applications, such as space exploration \cite{Meyer_2016}, may present economic routes to develop solar fertilizer technologies, but likely to be smaller in scale and are beyond the scope of this work. The three scenarios presented here are (1) inexpensive, robust solar fertilizer production at the scale of small farms in remote and undeveloped regions (2) solar fertilizer production integrated with existing infrastructure on larger farms in developed regions, and (3) high-tech solar fertilizer production coupled with production and distribution infrastructure of existing or emerging agricultural products. These scenarios present exciting opportunities to develop scalable decentralized solar fertilizer technologies with the potential for substantial positive impact on society, energy, and the environment.


\subsection{Considerations for dilute fertilizers}
\label{sec:dilute}

The centralized production of fertilizers along with the high purity of Haber-Bosch ammonia has driven the development of solid fertilizers with high weight percent nitrogen (35-85\%) to reduce transportation and storage costs. 
Utilization of solar energy is expected to produce fertilizers with nutrient densities substantially lower than the traditional Haber-Bosch process owing to the lower density of solar energy \cite{MacKay_2013} and the challenges with low efficiency and selectivity in photo(electro)chemical nitrogen fixation \cite{Skulason_2012,Singh_2017}. This is more similar to the biological production of nitrogen that occurs directly in the root system of the plants and results in relatively low local concentrations of fixed N in the soil, approximately 20 kg N/ha yr on average \cite{Smil_1999_2}. As we discuss in Section \ref{sec:targets} the required solar-to-ammonia efficiencies and nutrient concentrations are in principle surprisingly low ($<$1\%); however, these low-concentration fertilizer products differ substantially from existing fertilizers and present a number of challenges as well as some opportunities. Here we consider two varieties of dilute fertilizers: liquid fertilizers in aqueous solutions and solid fertilizers based on carbonaceous materials. These fertilizers have the potential to integrate well with existing agricultural practices, but will also require changes to existing fertilization practices.

Aqueous fertilizers are advantageous since plants require water as well as nutrients, and the process of simultaneously applying fertilizer and water is known as fertigation. Fertigation has formidable potential when coupled with solar fertilizer production since fertigation systems are able to deliver nutrients at a slow rate over time. This leads to a lower overall nutrient concentration relative to solid urea fertilizer, where much of it is washed away. This is particularly effective in areas with high rainfall such as Florida \cite{kadyampakeni_2015}. In tests fertigation has proven to be more effective than both traditional fertilizers and slow release fetilizers in producing growth in citrus trees \cite{Morgan2009}, and leads to higher NO3$^-$ concentrations in the top 15cm of soil \cite{Willis1991}. Tests in peach orchards showed improved fruit sizes with drip fertigation compared to conventional methods \cite{Bryla2005}. These practices may additionally lower by an order of magnitude (down to 13\% of current use) the amount of nitrogen fertilizer needed to achieve the same results as conventional methods \cite{kadyampakeni_2015}. 
Recommended concentrations of nitrogen in fertigation systems range from 50-200 ppm on a mass basis \needcite for most crops. This is in contrast to typical solid urea fertilizers that are 47 wt\%-N. This stark ($\sim$ 4 orders of magnitude) difference in the nutrient density by weight indicates that aqueous dilute fertilizers cannot be economically transported, meaning that aqueous dilute fertilizers are only viable with farm-scale production (see Sec. \ref{sec:decentralized}). This may lead to additional challenges with fertilizer storage, since the solar flux may not always align with crop nutrient needs. This would lead to the need for on-site storage tanks that would increase the footprint of the fertilizer production system, or electrochemical systems that can operate from the electricity grid to produce dilute fertilizers on demand. Another challenge is that fertigation relies on irrigation infrastructure for delivery. This may present a particular challenge for many smallholder farms in sub-Saharan Africa, where only around 7\% \hl{IFDC number from C. Dimkpa} of farms are equipped with irrigation \needcite. Nonetheless, these farms present a sizable initial market, and the prospect of combined fertilization/irrigation systems may be economically viable in already-irrigated farms in the developed world or favor investment in irrigation systems in the developing world.

Another approach to dilute fertilizers is to embed the nutrients with a carrier solid. In most current practical applications the form of nitrogen in fertilizer is urea, which comes as a solid that is dispersed over croplands. Solar fertilizer production could be coupled with adsorbents to uptake and concentrate the products, leading to a solid dilute fertilizer product. In this scenario the ammonia or nitrate products from the photo(electro) chemical reaction could be separated using a solid adsorbent such as activated carbon or biochar \cite{Gon_alves_2011}. This approach is advantageous since application of carbonaceous materials is a common agricultural practice \needcite, and adding adsorbent carbon to soil has been shown to provide many benefits for croplands including water retention, hydraulic conductivity, and resistance to soil erosion \cite{Li2018}. These changes manifest in the form of improved crop production \cite{Glaser2002}, although the magnitude of the improvements depend on the particulars of crop and soil type. Nonetheless, increases of 30\% in seed germination rate and 13\% in biomass production have been observed in woody plants \cite{Chidumayo_1994}. However, implementation of biochar fertilizers may be challenging, since the highest reported ammonia loading for biochar is 16 mg/g of NH$_3$ \cite{Gon_alves_2011}. This translates to 376 mmol/L on a volume basis, and would require 3.8 Mg ha$^-1$yr$^-1$ of activated carbon saturated with ammonia be applied to reach a nutrient loading of 50 kg-N/ha yr. \hl{How does this compare to a typical amount of carbon that would be applied in current agricultural practices?} An additional drawback is that producing high-surface-area carbon requires furnace temperatures above 400 $^\circ$C \cite{Lehmann2007}, which may present an engineering challenge in a low-resource setting. 
%These temperatures would be difficult to achieve without moderate capital investment, and are likely not possible in low resource environments. 
However, in some developed counties biochar facilities have been built and proven profitable \cite{McHenry2009}, and in others would be profitable with a moderate carbon tax \cite{Galinato2011}. This suggests that integration of solar fertilizer facilities with production facilities for carbonaceous soil additives may be a promising strategy, although considerable research is required to determine the efficacy of carbonaceous dilute fertilizers in real agricultural settings.

One enticing possibility for both aqueous and carbonaceous dilute fertilizers is the prospect of improved nutrient management. Currently, the fixed nitrogen in fertilizers is not utilized efficiently, with 20-50\% being lost to leaching or vaporization \cite{Smil_1999_2, Naz_2016}. Leached fertilizer then enters waterways, leading to hypoxic regions in oceans (called dead zones), eutrophication in lakes and rivers, and groundwater contamination \cite{Diaz2008,Conley_2009,Shindo_2006}.  Pollution of this kind is acute in the Gulf of Mexico and the North Sea, due to the intensive agricultural practices of the developed nations upsteam \cite{Diaz2008,Conley_2009}. The vaporization of nitrogen fertilizers can also have deleterious effects on the environment by releasing NO$_x$ compounds, damaging the protective ozone layer \cite{Ravishankara_2009}. The highly concentrated fertilizers responsible for this pollution release nutrients too rapidly for plant uptake, with researchers estimating they are only used at an efficiency of 20-35\% \cite{Naz_2016}. This can have negative effects on the plants themselves, damaging root systems in areas of high concentration \cite{Morgan2009}. The most common strategy for mitigation of these effects is the development of coatings that aid in controlled release of nutrients and enhance uptake efficiency \cite{Naz_2016} \hl{Would be good to get some more examples to cite here}. While these slow-release fertilizers improve performance, the use of dilute fertilizers offers a different approach in which nutrients are delivered at a controlled rate, and in smaller overall amounts \cite{kadyampakeni_2015}. This would enable matching N supply with crop demand. However, substantial additional research into agronomics and plant nutrition is required to determine the potential of this strategy and identify the optimal nutrient concentration and application profile. If dilute fertilizers are applied through fertigation, there is an opportunity to fertilize crops each time they are irrigated, controlling exactly the timing of nitrogen addition. Nitrogen-saturated biochars also hold promise, as these could release nitrogen slowly over a growing season, more effectively resisting leaching than solid urea with the added benefit of improved retention of P and K based fertilizers in the soil through higher ion exchange capacity \cite{Glaser2002}. In addition to improved nutrient management, dilute fertilizers are inherently safer since they will be less corrosive and more difficult to convert into explosives. These factors suggest that further research into the utility and effectiveness of dilute fertilizers is relevant to the field of solar fertilizers.


\section{Solar fertilizer production processes}
\label{sec:approaches}

The prospect of solar fertilizers share much with the well-studied approaches to solar hydrogen and solar fuel production. For example, both require photon absorption, catalysts for the reaction, efficient transfer of energy from the absorber to the catalyst, and materials that are stable under operating conditions. However, there are also some key differences since solar fertilizers must integrate with agricultural infrastructure, and the products are different in chemistry and concentration. In this section we examine three key aspects of solar fertilizer process design: solar capture, reaction and catalysis, and separations. We focus primarily on aspects unique to solar fertilizers, and refer to numerous reviews on solar hydrogen or fuels for additional considerations \cite{McDaniel_2010,Pinaud_2013,Highfield_2015, Shaner_2016, Montoya_2017}. 

\subsection{Solar capture}
\label{sec:solarcapture}

The solar fuels community has identified two basic strategies for conversion of solar to chemical energy: direct capture of photons through photochemistry (photocatlaysis), or indirect capture through photovoltaics coupled to electrochemistry (PV-electrolysis)\cite{McDaniel_2010,Highfield_2015}. Photoelectrochemistry, whereby electrical bias is applied during solar capture represents a third hybrid (indirect+direct) approach for solar-fuel production. There has been considerable debate and analysis regarding the efficiency of each approach for fuel production\cite{Montoya_2017,Daviseaas9793,Lewis_2016,Herron_2015}, and while there is no clear consensus, indirect capture has received considerable attention for the production of hydrogen. This has largely been driven by the goal of achieving solar-to-hydrogen efficiencies which approach 20\% and cost which approach 5.7 \$/kg \needcite; indeed the goal of 20\% efficiency has been achieved by numerous systems \needcite \hl{refs. 8-10 in the Montoya review}. Solar fuels technologies are typically envisioned to operate at large scales in relatively centralized industrial production facilities \cite{Shaner_2016} \needcite. Yet, in the case of solar fertilizers there is a strong motivation for technologies that operate in decentralized locations or at an agricultural site as discussed in Sec. \ref{sec:decentralized}. Indirect solar capture requires a relatively high level of technological sophistication since solar capture arrays, electrochemical reactors, and associated electrical connections and controls must be maintained, and fertilizer products must be separated from the electrolyte. Furthermore the low areal energy density of photovoltaics \cite{MacKay_2013} coupled with the need for separate solar capture, fertilizer synthesis, and separations facilities will lead to a relatively large footprint for indirect capture. This suggests that PV-electrolysis approaches are best suited to semi-centralized solar fertilizer production, or on-site production at large farms in developed regions (see Sec. \ref{sec:decentralized}). Nonetheless, photovoltaic technology is well-established, and efficiencies of 10-20 \% are typical. This leads to a required electrical-to-ammonia efficiency of $\sim$ 1\%, which is relatively low and has been reported at the lab scale for state-of-the-art ammonia electrocatalysts.\cite{Liu2018,Qiu_2018,Song_2018,Zhang_2018,Luo_2018} Further, electrochemical fertilizer production can be integrated with an electrical grid (though the fertilizers resulting from grid-based electricity cannot technically be considered ``solar fertilizers''), providing reliable yields even in periods of no sunlight, and the use of high current densities can enable the production of higher concentrations of fixed nitrogen. In addition, electrochemical technologies have been demonstrated at scale, including the chloroalkali process, water hydrolysis, and hydrogen fuel cells\cite{Burney1993} \needcite, and many of these technological developments could be applied to semi-centralized solar fertilizer production processes.

%Nonetheless, the efficiencies of electrochemical nitrogen reduction are generally higher than direct photochemical conversion, and integration with the electrical grid provides dispatchability in times of low solar flux (though the fertilizers resulting from grid-based electricity cannot technically be considered ``solar fertilizers'').

% metrics addressed: efficiency, technological sophistication, farm footprint, dispatchability, (safety), (materials constraints/requirements)

The alternative approach of direct capture and photocatalytic conversion through a single material or integrated device has also been explored for solar fuel production \cite{Montoya_2017, Lewis_2016, Pinaud_2013}, and some technoeconomic analyses suggest that particle bed photocatalytic systems will lead to the lowest costs, although the potentially explosive product mixtures present technical challenges \cite{Pinaud_2013}. In the case of solar fertilizers this safety concern is alleviated since low-concentration products are expected (see Sec. \ref{sec:dilute}). Direct solar capture systems contain few if any moving parts, driving down the expected costs of maintenance and installation and making them better suited for decentralized fertilizer production at small scale farms in developing regions. However, production rates in direct capture are directly proportional to the solar flux, leading to uncertainty in production capacity. This uncertainty can potentially be mitigated through storage, though this will increase the footprint of the solar fertilizer production process, and/or by identifying regions where the solar flux is high such as sub-Saharan Africa or India (see Fig. \ref{fig:usemap}). Another challenge is that the highest reported efficiencies for direct photocatalysis are relatively low (0.01 \%). In general, the materials constraints for direct capture are more stringent since the same material must act as an absorber and a catalyst, or the interfaces between the two materials must be carefully engineered \cite{Montoya_2017}. The constraints are even more severe when cost is considered, since materials containing rare elements or requiring expensive processing are unlikely to be viable in a low-cost design. Yet, many reported catalysts are based on earth-abundant materials such as TiO$_2$, Fe$_2$O$_3$, or carbon \cite{Medford_2017} \needcite \hl{we should include some direct references to original sources here}, and required efficiencies are expected to be $<$ 1\% (see Sec. \ref{sec:targets}) indicating that inexpensive, low-efficiency photochemical reactors similar to those used in air purification may be viable \cite{Birnie2006, Bhatkhande_2001,Parkin_2005}.

In the case of direct capture the band gap and band edge alignment of the material must be optimized along with the catalytic performance. The optimal band alignment will depend on the absorber configuration (single vs. dual) and the over-potential required for the oxidative and reductive half-reactions. Substantial effort has been dedicated to the question of optimal band configuration for solar fuel production, resulting in several modeling frameworks \cite{Seitz_2014, Seger_2016}. These tools can be easily adapted to optimize band configuration and identify performance limits for solar fertilizers \cite{Medford_2017}; however, knowledge of the half-reactions and catalytic over-potentials is required. There are still open questions regarding the relevant half-reactions and catalytic mechanism for photocatalytic nitrogen fixation \cite{Davies1995,Medford_2017,Comer_2018,Comer_JACS}. Resolving these fundamental questions is of critical importance for the practical optimization of solar fertilizer technology.

\subsection{Reaction and catalysis}
\label{sec:reaction}

%% Figure - band edges of TiO2 vs NH3/NO/C formation potentials

The chemical conversion of dinitrogen is at the heart of solar fertilizer production, and the extremely strong nature of the N$\equiv$N triple bond requires a catalyst to accelerate the kinetics of nitrogen dissociation, especially at benign temperatures. The vast majority of photo(electro)chemical approaches to nitrogen fixation have focused on the chemical reduction of nitrogen to ammonia. The electrochemical nitrogen reduction reaction (NRR) has been increasingly studied, and has a redox potential of \hl{-0.05?} V vs. RHE and requires an overall applied potential of 1.17 V when coupled with the oxygen evolution reaction based on standard thermochemical data \cite{Medford_2017}. Photo(electro)chemical production of ammonia is a promising for fertilizer production since many existing fertilizer products utilize ammonia \needcite, and numerous catalysts have shown ammonia synthesis activity in experiments \needcite. However, the proximity of the redox potential or nitrogen reduction and hydrogen evolution presents a fundamental challenge in nitrogen reduction since the hydrogen evolution reaction is typically faster resulting in low selectivity \cite{Skulason_2012,Singh_2017}. Furthermore, nitrogen reduction is typically performed under anaerobic conditions to avoid competition with O$_2$ adsorption or reaction \needcite, a situation that would necessitate air separation in a practical setting. Another challenge in nitrogen reduction is the fact that oxygen evolution is typically utilized as a half-reaction. Oxygen evolution catalysts typically exhibit large overpotentials of $\sim$ 0.4 V, and are often based on rare materials, presenting challenges for process efficiency and scalability. Nonetheless, the best reported catalyst for nitrogen reduction is based on earth-abundant carbon and exhibits an electrical-to-ammonia efficiency of 5\%, suggesting that practical routes to electrochemical nitrogen reduction are feasible.

A less-explored alternative is direct oxidation of nitrogen to nitrate products. Nitrate production is more thermodynamically favorable than ammonia synthesis. The production of NO is the most thermodynamically challenging step, and requires only 0.45 V of applied potential when coupled with the oxygen reduction half-reaction, although more extreme oxidizing potentials of 1.68 V vs. RHE are required based on standard thermodynamic data \cite{Chen_2018,Medford_2017, Comer_2018}. Nitrate-based fertilizers are also common, and some crops are able to utilize nitrates more efficiently that ammonium, although nitrates are also more volatile and prone to leaching \needcite. One key advantage of nitrogen oxidation is that it can occur directly in air, since oxygen is a reactant, and competition with hydrogen evolution is not an issue. Despite these promising advantages, there are considerably fewer reports of photocatalytic nitrate formation \needcite, and electrocatalytic nitrogen oxidation has not been demonstrated on any catalysts. This indicates that catalyst development for nitrogen oxidation will require more effort. 

Another possibility is coupling nitrogen reduction with carbon-based reactions. A simple example is the use of sacrificial reagents such as alcohols that reduce the overall driving force needed for nitrogen reaction by avoiding the need for the kinetically-challenging oxygen evolution reaction. This practice is common in photocatalysis, and indeed methanol and ethanol have been reported to increase photocatalytic ammonia yields \needcite. Recent work has shown that surface-bound carbon species play a role in photocatalytic nitrogen fixation on TiO$_2$, suggesting an alternative route through which hydrocarbon species may accelerate nitrogen fixation. Hydrocarbon reactants are abundant in an agricultural scenario, and very early work bases on illuminated compost experiments suggested that photocatalytic reactions involving hydrocarbons and nitrogen induced improved nutrient content in soils \needcite. An alternative and overly-abundant carbon-based feedstock is CO$_2$, and simultaneous photo(electro)chemical reduction of CO$_2$ and N$_2$ may enable urea formation \needcite. The combination of carbon and nitrogen chemistry opens a rich range of possibilities, many of which have received little to no scientific attention. 

\subsection{Separations}
\label{sec:separation}

The chemical separations required to convert the effluent of a reaction to a fertilizer are also of critical importance to advance solar fertilizer technology. In the case of solar fuels this is less critical, since many fuels like hydrogen are gaseous and easily separable \needcite. However, ammonia, nitrates, and urea are all highly water soluble. This is a particular problem in the case of electrochemical nitrogen fixation, where the product must be separated from the electrolyte. Further, if the process is not resistant to oxygen or other common environmental contaminants then an air separation or purification unit will be necessary. In addition to the chemical separation it may also be necessary to separate the catalyst from the solution, for example in the case of slurry photoreactors. These separations are a critical consideration for decentralized fertilizer production since high capital investment and expert operation may be required, which would not be feasible at the scale of a small or even a relatively large farm. We briefly discuss the key separations challenges for solar fertilizers: separation of nitrogen from air, upgrading the concentration of products, separation of products from the electrolyte, and separation of the catalyst from the electrolyte. The possible application of absorption, distillation, and/or membrane separation technologies are considered for each case.

Many photo- and electrochemical processes for nitrogen fixation are based on a pure nitrogen feedstock. For example, oxygen has been shown to inhibit photocatalytic nitrogen fixation over the commonly-used TiO$_2$ catalysts \cite{Hirakawa_2017}, and high-purity nitrogen is typically used in electrochemical tests \cite{Song_2018} \hl{We can cite others here. Has anyone studied O2 inhibition for electrochem?}. The need for air separation presents a critical challenge for farm-scale fertilizer production. The most common air separation processes are based on cryogenic distillation, which is energy intensive (\hl{energy estimate}) and requires large scales ($>$ 2000 kg-N/h) \needcite. Cryogenic separation would not be economically feasible, even at the scale of a large farm, suggesting that semi-centralized production is the most viable production scenario if high-purity nitrogen is required. Other air separation technologies such as pressure-swing adsorption are more viable at smaller scales, but the purity of the resulting nitrogen is typically lower \needcite. Other approaches based on liquid sorbents, or even water, can passively enrich air by exploiting differences in nitrogen/oxygen solutbility \needcite. The need for air separation can also be mitigated altogether by designing processes that are directly compatible with air.

The concentration of fixed nitrogen in the product stream may also need to be upgraded to produce viable fertilizer products. Solar fertilizer products are generally expected to be more dilute, as discussed in Sec. \ref{sec:dilute}. Nonetheless, strategies to separate or concentrate the fixed nitrogen product may be required even for dilute fertilizers, particularly if production is semi-centralized. Separation of aqueous ammonia is challenging due to the strong hydrogen bond between water and ammonia, and is complicated by the effect of pH since ammonia is more soluble in acidic solutions \needcite. Most research in separating ammonia from water has been in the field of wastewater treatment where steam stripping from basic solutions has been shown to efficiently remove trace ammonia, and flash distillation is used for higher concentrations \needcite. However, these processes are optimized to reduce ammonia concentration rather than increase it, and are capital and energy intensive. Another possibility is to capture energy from the infrared region of the solar spectrum and use the resulting heat for passive distillation. An alternative approach is the use of adsorption for separation. This is particularly promising if the adsorbent itself acts as a part of the fertilizer, for example if biochar is used as an adsorbent as discussed in Sec. \ref{sec:dilute}. This removes the need for an energy-intensive desorption cycle, although it is critical that the absorbent release nutrients when placed in the soil. Ideally, the need for upgrading can be mitigated by discovery of catalysts that are both active and selective for nitrogen fixation, and through design of processes that result in effluents with high concentrations of ammonia or nitrates.

Another consideration is that photo(electro)chemical processes often require electrolytes to provide electrical conductivity and control the pH. For example, some of the highest reported electrochemical ammonia formation rates are based on Li-based electrolytes \cite{Song_2018}. In general the effect of electrolytes on plant growth are unknown, and electrolytes that contain metals such as Li can be costly, indicating that separation of electrolytes may be required. This could potentially be achieved relatively efficiently via precipitation or membrane-based separation processes, though electrolyte recovery has received relatively little attention \needcite. An alternative approach is to seek electrolytes that are abundant and non-toxic, such as NaCl, or utilize electrolytes such as KOH and Na$_2$H$_2$PO$_4$ that provide an additional source of P and K nutrients. Research into the role of electrolytes and pH on soil fertility and plant nutrition can identify optimal or acceptable ranges for dilute aqueous fertilizers. This will enable design of photo(electro)chemical processes where electrolyte selection minimizes or removes the need for additional separations.

The final role of separation is extraction of the catalyst from the electrolyte. This is only required in the case of particle slurry photocatalytic reactors, or homogeneous photo(electro)catalysts. Membranes or sieves present an efficient opportunity for removing catalyst particles, since the size of particles will be substantially larger than the molecules in the electrolyte, reactants, or products. Another possibility is to separate the fixed nitrogen products directly from the electrolyte via adsorption, effectively immobilizing the products on the absorbent. In the case of homogeneous catalysts the removal of the catalyst is substantially more challenging. Separation of homogeneous catalytic complexes has been the subject of research in many other contexts, and generally requires sophisticated and energy-intensive processes that maintain the integrity of the catalyst complex. The need for this separation can be mitigated by reactor designs with supported catalysts, and the use of solid catalyst materials.


\section{Preliminary Performance Targets}
\label{sec:targets}

There has been a substantial recent increase in photo(electro)chemical nitrogen fixation research, yet there are no clear targets for how efficient these processes need to be to enable practical impact. Further, the metrics typically used to assess the performance of catalytic materials are not standardized or clearly linked to solar fertilizer yield. Substantial effort was devoted to identifying standardized tests and benchmarking procedures for photo(electro)catalytic water splitting \cite{Chen_2010, McCrory_2015, Smith_2016, Bligaard_2016}, many of which are relevant for solar fertilizers. In this section we propose several metrics that capture the photon absorption, reaction, and separation performance: solar-to-nutrient efficiency, nitrogen fixation rate, energy per nitrogen fixation, and energy per utilizable nitrogen. Performance targets for these metrics are identified, and relevant testing conditions such as solar spectrum, operating current density, oxygen content, and nutrient concentration are discussed. These targets are not meant to be authoritative, but rather provide guidelines for catalyst development and fertilizer testing.

Assessing the photon absorption performance in the case of direct absorption and photocatalytic conversion is best assessed by the efficiency of converting solar energy to the chemical energy of the nitrogen nutrients in the fertilizer. The chemical energy of nutrients varies between ammonia, nitrates, and urea, and the required nutrient load also varies depending on the crop and agricultural region, making it difficult to identify an exact target for solar-to-nutrient efficiency. An order-of-magnitude estimate for the areal energy density required for fertilization is obtained by assuming the average nutrient density of 50 kg-N/(ha yr) is provided by ammonia-based fertilizers:
\begin{equation}
\mathrm{
50 \frac{kg_N}{ha . yr} \times \frac{10^3}{14} \frac{mol_{NH_3}}{kg_N} \times \frac{667}{2} \frac{kJ}{mol_{NH_3}} \times \frac{1}{10^4} \frac{ha}{m^2} \times \frac{1}{3.15e7} \frac{yr}{s} = 3.77 \frac{mW}{m^2}
}
\end{equation}

An initial estimate of necessary efficiency can be obtained by assuming 8 hours of full sunlight per day and 1\% of arable land dedicated to solar capture \cite{Medford_2017}. In this case the average solar flux is 333 W/m$^2$ and the corresponding required solar-to-ammonia efficiency is $\sim$0.1 \%. Data on actual average solar fluxes reveals that they vary from 120 - 280 W/m$^2$ depending on latitude \cite{MacKay_2013}, and there is also considerable variability in the nutrient load required, ranging from 15-200 kg-N/m$^2$ depending on a myriad of factors including crop and soil type \needcite. Based on these estimates the required solar-to-ammonia efficiency may range from 0.05 - 1.25 \% depending on solar flux and required nutrient load. These estimates assume that 1\% of arable land is dedicated to solar capture, and will vary linearly with the percentage of land available. We propose that 1\% is a relatively conservative number, corresponding to 100 m$^2$/ha or roughly 6 typical solar panels per hectare \needcite. The solar-to-ammonia efficiency of electrochemical processes can be obtained from the energy-to-ammonia efficiency. This rarely reported metric can be multiplied by the efficiency of solar photovoltaics ($\sim$ 20\%) to obtain a solar-to-ammonia efficiency. Notably, solar-to-ammonia efficiencies as high as 0.02 \% have been reported for nitrogen fixation photocatalysts \cite{Hirakawa_2017}, and the highest reported energy-to-ammonia efficiency for an electrocatalytic process is 5\% \cite{Song_2018}, corresponding to a solar-to-ammonia efficiency of 1\%. These promising results suggest that practically relevant solar-to-nutrient efficiencies are likely attainable for both direct and indirect photo(electro)chemical nitrogen fixation. \needcite

The rate of nitrogen fixation is the most often reported metric for catalyst performance; however, there is not a standard for how the rate is normalized. Photochemical rates are often reported as mass of ammonia per unit-mass of catalyst, while electrochemical rates are often normalized per unit area of the electrode. Key quantities such as catalyst loading and illumination area are needed to effectively compare the rates, yet these are not always reported. In the case of photochemical nitrogen fixation the rate normalized to illumination area is critical, although this is also proportional to the solar-to-nutrient efficiency. For electrochemical nitrogen fixation the rate will be dependent on the applied potential, and proportional to the product of the current efficiency and Faradaic efficiency toward fixed nitrogen products. In the case of nitrogen reduction the Faradaic efficiency typically depends on the applied potential, and decreases at high current densities, leading to an optimum operating potential. This leads to rates and efficiencies that are reported at different operating potentials for different catalysts. In the case of solar fuels the overpotential for oxygen evolution has been defined as the potential at which the current is equal to 10 mA/cm$^2$, equivalent to 10 \% solar-to-fuel efficiency. For solar fertilizers the necessary current will be substantially lower, due to the substantially lower solar-to-ammonia efficiency requirements. For simplicity, we will assume that a 1\% solar-to-ammonia efficiency is required, resulting in a relevant current density of 1 mA/cm$^2$, substantially higher than many reports where current densities are in the range of $\mu$A \needcite. Another consideration in reporting rates is the fact that products other than ammonia such as nitrates or urea are also fertilizers. For this reason reporting rates in $\mu$g-N/hr cm$^2$ (as opposed to $\mu$g NH$_3$) may be preferable for comparing yields between different fixed nitrogen products.

An alternative approach to comparing catalyst performance across different reactions and photo(electro)chemical approaches is quantification of the energy input required to produce one mole of fixed nitrogen product. This molar energy density metric can be computed for photochemical, electrochemical, and thermochemical nitrogen fixation, as well as ammonia, nitrates, urea, or other fixed nitrogen products. For example, the energy requirement for Haber-Bosch is \hl{???}, while the best reported electrocatalytic and photocatalytic molar energy densities are  \hl{???} and \hl{???} respectively. The molar energy density is closely related to the energy efficiency, and a target corresponding to 50 kg-N/ha-yr at 200 W/m$^2$ illumination is \hl{???}.

A related metric with more practical impact is the energy density required to generate one mole of utilizable nitrogen from ambient conditions. This metric is related to the separation efficiency as well as the catalyst performance, and is a function of the overall process. This makes it more difficult to quantify, but considering the energy required for separations is necessary when comparing different catalytic processes. The two key separations are purification of nitrogen from air, and separation of products from the electrolyte. While a detailed analysis of the necessary energy requires knowledge of the process, it is possible to establish estimates based on thermodynamic limits. In the case of air separation this leads to \hl{...}. 

The conditions under which the metrics are measured are also important. For example, solar efficiency should be measured with an AM1.5 solar simulator rather than xenon or mercury lamps with high UV content. Further, the rates and efficiencies should be measured in air to assess the influence of oxygen. The performance can be further probed by running the catalyst until the concentration of nutrients approaches a practical value. Studies on fertigation have reported nutrient densities as low as 50ppm, or 0.005 mol \%. The concentration of fixed nitrogen products in the effluent of lab-scale tests is rarely reported, but may also have an impact on the performance since rates may be suppressed as the concentration increases. Performing experiments that result in nutrient concentrations approaching 50ppm with air as a feedstock under simulated solar illumination will provide convincing evidence that a catalyst is capable of distributed solar fertilizer production.

\section{Conclusions}

Solar fertilzers present an exciting opportunity to directly capture diffuse solar energy and convert it to chemical energy that can be applied at or near the point of production. The technology falls at the complex nexus of energy and agriculture, and substantial additional research is needed to establish the most promising approaches and demonstrate the technology. This work grapples with some initial considerations from the perspective of agronomics and photoelectrochemistry, and identifies some preliminary metrics that will aid in the development and deployment of solar fertilizer technologies. Specific metrics and testing conditions are presented to identify the targets that may enable solar fertilizer technology. The metrics and considerations presented draw on a range of expertise in the diverse fields of agronomics, photoelectrocatalysis, chemical separations, and process systems engineering, and provide a starting point for further development of solar fertilizer technologies. There are many possible routes forward for this nascent field, and identifying the most promising will require a diverse range of technical, social, and economic considerations. However, the vast potential impact of solar fertilizers on the growing problem of world hunger makes this challenging endeavor worthwhile.


