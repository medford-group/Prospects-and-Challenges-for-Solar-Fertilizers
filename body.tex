\section{Introduction}

% Motivate need for new process
The Haber-Bosch process for thermochemical synthesis of ammonia from nitrogen and hydrogen transformed the global fertilizer industry and was a critical enabler of the continued expansion of human populations \cite{Smil_1999}. The process is an impressive feat of modern chemical engineering, producing around 150 million tons of ammonia per year at a thermodynamic efficiency of up to 70\% \cite{Schloegl_2003,Schiffer_2017}. However, the process also has downsides. The scale of the process leads to massive energy consumption of 2.5 exajoule per year, and the hydrogen feedstock is typically obtained via methane reforming, leading to a tremendous carbon footprint of 340 Mt of CO$_2$ equivalent per year; this is the highest impact of any commodity chemical production \cite{Schiffer_2017}. Furthermore, the high temperatures ($\sim$700 K) and pressures ($\sim$100 bar) lead to substantial capital cost and favorable economies of scale, driving highly centralized production. This is in contrast to the decentralized use of ammonia-based fertilizers, which results in substantial transportation costs and additional emissions. This is particularly impactful in remote locations such as sub-Saharan Africa, where soils are often nutrient-limited due to lack of access to fertilizers\cite{Gilbert_2012, Mueller_2012}. The opposite problem of over-fertilization also has a substantial environmental impact in more developed nations due to the periodic application of highly concentrated fertilizers that cause nitrate pollution in groundwater through leaching. Finally, the intense conditions and reactive nature of concentrated ammonia lead to safety and national security concerns, as evidenced by massive explosions at fertilizer plants and the common use of fertilizers in makeshift explosives.

% Motivate solar fertilizers
One possible strategy to overcome these disadvantages is decentralized production of fertilizers using solar energy. These ``solar fertilizers'' could harness solar energy, nitrogen, and water/oxygen from the air to produce lower-concentration ammonia- or nitrate-based fertilizers at or near the point of use. This is advantageous from the perspective of solar energy capture since the intermittent energy is directly captured in a storable product that can be utilized near the point of production, avoiding issues with electricity storage and transport \cite{MacKay_2013}. There are also advantages from an agricultural perspective, since inexpensive feedstocks and reduced transport costs may significantly improve access to fertilizers in remote and developing regions, and the use of lower-concentration fertilizers may also enable novel strategies of nutrient management that can reduce groundwater pollution. Further, low-concentration fertilizers produced at ambient conditions are inherently safer both from the perspective of the process and the product. 

% Motivate need for collaboration with agriculture/fertilizer
Solar fertilizers hold substantial promise as a route to solar energy capture and sustainable agriculture, but there are also considerable challenges. One critical and obvious challenge is in the development of a viable strategy for efficiently using solar energy to dissociate the strong dinitrogen triple bond at ambient conditions. Nitrogen fixation at ambient conditions is a holy grail of chemistry, and has been the subject of considerable research in homogeneous catalysis, enzyme catalysis, and bioengineering, yet no viable strategies have emerged due to issues with low conversion and/or stability under realistic conditions \cite{Vicente_2017}. %need lots more citations here.
More recently there has been a surge of interest in photo- and electrocatalytic nitrogen fixation by heterogeneous catalysts \cite{}. This route is particularly interesting from the perspective of solar fertilizers since photoelectrochemical systems have the potential to scale relatively easily and have been the subject of considerable research in the solar fuels community. However, further work is needed to improve the yield and efficiency of photo- and electrochemical nitrogen fixation. %In this work we focus on photo(electro)chemical routes to solar fertilizer production and seek to identify targets and strategies for future research.

Solar fertilizers are also expected to differ significantly from traditional fertilizers, opening a range of additional agronomic challenges. One key difference is that solar fertilizers are expected to have considerably lower fixed nitrogen concentrations, owing to the fact that photo(electro)chemical nitrogen fixation efficiencies are unlikely to compete with the 70\% efficiency of the Haber-Bosch process. Separating the ammonia would be energy intensive and likely require centralized facilities, potentially mitigating the advantages of decentralized solar fertilizer production. Direct utilization of dilute fertilizers would avoid or reduce the amount of energy needed for separation/concentration, and may also enable more controlled nutrient management. However, this represents a paradigm shift in agricultural practice, and considerable effort is needed to understand how dilute solar fertilizers can be integrated with agricultural systems. These considerations will also inform the development of the photo(electro)catalytic processes for solar fertilizer production, and hence should be considered in parallel.

In this work we identify key considerations and performance targets for the photo(electro)chemical production of dilute solar fertilizers from the perspective of catalysis and agronomics. Some specific advantages and disadvantages of decentralized dilute fertilizer production are outlined and the potential agronomic impacts of varying levels of decentralization are examined. Initial quantitative performance targets are identified based on back-of-the-envelope calculations and possible agricultural scenarios. Finally, the tradeoffs between photo- and electrochemical routes to solar fertilizers are briefly discussed, and some basic reactor design concepts are outlined. We hope that these considerations will serve as a foundation and guide for future research in the development of photo(electro)chemical processes for solar fertilizer production.
%Key considerations are outlined, and several quantitative targets are identified. These targets are preliminary due to the nascent stage of solar fertilizer research, but will provide a starting point for future refinement as the field develops. 

\section{Decentralized Dilute Fertilizers}

The price of fertilizer is controlled by a complex interplay of geopolitical and economic factors. A detailed analysis is beyond the scope of this work, but we briefly introduce some key concepts. The cost of fertilizers can be broken down into production, transportation, and storage. The production cost of fertilizer is controlled primarily by the cost of natural gas used to produce hydrogen for the Haber-Bosch process, and hence varies with location and geopolitical factors. The cost of transportation is highly variable. Transportation costs as low as ??? \% in the U.S. where ammonia pipelines connect fertilizer production facilities along the Gulf of Mexico to the agricultural centers in the Midwestern states, or as high as 25\% in landlocked sub-Saharan African countries such as Uganda where fertilizers must be transported via ships to a port and subsequently via trucks to agricultural centers. For countries without large-scale fertilizer production it is common to separate transportation costs into domestic and international, since international transportation is more expensive and sensitive to political factors such as tariffs. Finally, the cost of storage must also be considered, although it is generally less variable than transportation or production (??? no idea if this is true). Figure ??? shows a comparison of the percentage of fertilizer cost arising from each category in 2? 3? representative countries, illustrating the considerable economic impact of fertilizer transportation in the developing world. (I am envisioning 2-3 pie charts with size corresponding to total fertilizer cost and breakdowns of production/transportation/storage contributions)

The total price of fertilizer is not derived directly from its components. For example, the price of fertilizer in Thailand (\$ 287/ton) is roughly half the price of fertilizer in Mozambique (\$ 567/ton), but this difference cannot be attributed directly to any single category. One critical factor that controls this overall price is economy of scale. Larger agricultural markets such as Asia are able to more effectively distribute fixed costs of transportation across more units of fertilizer, translating to lower prices at the farm scale. This scale-up is not possible in less developed markets for a variety of reasons including port capacity and poor transportation infrastructure. Less developed markets are also subject to more uncertain demand owing to lack of access to capital by smallholder farmers and unpredictable implementation of government subsidies. Overall, these factors lead to the perverse situation in which fertilizers are most expensive in the poorest places where the need is greatest. This is a key factor in the distressing fact that world hunger is currently increasing, with over 800 million people suffering from undernourishment as of 2016 \cite{FAO_2017}.

Notably, many of these economic and geopolitical factors could be alleviated by decentralized fertilizer production from renewable resources. Lack of dependence on natural gas would reduce volatility in production costs, and producing fertilizer at or near the point of use would reduce transportation costs and reduce the price dependence on economies of scale. 

* Advantages of distributed vs centralized fertilizer production

* Advantages (e.g. safety) and challenges (e.g. volume) of dilute fertilizers

\section{Targets from Agronomics}
* Minimum fixed-N concentration for fertilizer (1 mol% ?)

* Minimum required solar-to-ammonia efficiency (0.1 %?)

* Expected current densities (analogy to 10 mA/cm$^2$ for solar fuels)

\section{Approaches for Solar Capture}

* Tradeoffs with photo/electrochemical

* Direct capture

* Indirect capture

   - toxicity of electrolytes


\section{Quantification of Ammonia}
* Interference of electrolyte or sacrificial reagents

* Contamination from experimental setup or gasses

* Discussion of isotopic labeling and other best practices

\section{Conclusions}


